Onthoud
\begin{align*}
    \sum_{k=c}^{\infty} r^k &= \frac{r^c}{1-r} \text{ als $|r|<1$}\\
    \sum_{k=1}^{\infty} \frac{1}{k^2} &= \frac{\pi^2}{6} \\
    \sum_{k=0}^\infty \frac{x^k}{k!} &= e^x \\
\end{align*}
Uit de eerste kunnen we andere sommen afleiden.
bijvoorbeeld
$\displaystyle \sum_{k=0}^{\infty} kr^k: $
\begin{align*}
    \sum_{k=0}^{\infty} r^k &= \frac{1}{1-r} \text { differenti\"eren aan beide kanten } \to \\
    \sum_{k=1}^{\infty} kr^{k-1} &= \frac{1}{(1-r)^2} \\
    \sum_{k=0}^{\infty} kr^k &= \frac{r}{(1-r)^2} \\
\end{align*}
Op dezelfde manier $\displaystyle \sum_{k-1}^\infty \frac{1}{k} r^k$, bijvoorbeeld $\displaystyle \sum_{k=1}^{\infty} \frac{1}{k2^k}$:
\begin{align*}
    \sum_{k=0}^{\infty} r^k &= \frac{1}{1-r} \text { integreren aan beide kanten } \to \\
    \sum_{k=0}^{\infty} \frac{1}{k+1} r^{k+1} &= -\log_e (1-r) \\
    \sum_{k-1}^\infty \frac{1}{k} r^k &= \log_e (\frac{1}{1-r})
\end{align*}
Eindige sommen kunnen we nu ook, \textbf{deze formule geldt algemener voor $r\neq 1$} maar dan is de afleiding natuurlijk anders.
\begin{align*}
    \sum_{k=c}^n r^k &= \sum_{k=c}^\infty r^k - \sum_{k=n+1}^\infty r^k
\end{align*}