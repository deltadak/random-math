Om momenten makkelijk uit te rekenen kun je gebruik maken van kansgenererende functies in het geval van een discrete verdeling, of Laplacetransformaties voor een continue verdeling.

De kansgenererende functie is, voor een random variabele $X$,
\[
    P(z) = \E{z^X} = \sum_{k=0}^\infty \P{X=k} z^k.
\]
Deze kun je uitrekenen of vinden in een statistisch compendium.
Als je de afgeleide hiervan neemt, zie je dat
\[
    \frac{\d}{\d z} \E{z^X} = \E{Xz^{X-1}}
\]
dus
\[
    \E{X} = \eval{\frac{\d}{\d z} P(z)}_{z=1}.
\]
Evenzo,
\[
    \frac{\d^2}{\d z^2} P(z) = \E{X(X-1)z^{X-2}} = \E{X^2} - \E{X}
\]
dus
\[
    \E{X^2} = \eval{\frac{\d^2}{\d z^2} P(z)}_{z=1} + \E{X}.
\]
Enzovoorts voor het $k$-de moment.

\paragraph{Voorbeeld}
Zij $X \sim \Poi(\lambda)$, dan is $P(z) = e^{\lambda (z-1)}$ dus
\[
    \E{X} = \eval{\lambda e^{\lambda (z-1)}}_{z=1} = \lambda
\]
en
\[
    \E{X^2} = \eval{\lambda^2 e^{\lambda (z-1)}}_{z=1} + \E{X} = \lambda^2 + \lambda .
\]

Voor Laplacetransformaties geldt
\[
    \phi(s) = \int_0^\infty e^{-st} f(t) \d t = \E{e^{-sX}}
\]
en de rest kan op dezelfde manier worden afgeleid.
